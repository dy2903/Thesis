% !Mode:: "TeX:UTF-8"

\chapter{TIADC系统失配误差模型的研究}
		
	本章\ref{sec:principleOfTIADC}小节中对TIADC有效性的推导建立在各通道工作在同样的时钟并且相邻子ADC的相位严格可控,同时通道的偏置、增益等参数完全相同的情况下。由于每一个通道都是一个单独的ADC,所以每一个通道都独自含有一些误差。但同时因为它们又构成一个完全并行的系统,参数间的失配同样会产生额外的误差。所以TIADC系统中偏置、增益、时钟偏置误差以及其他各种非线性误差会对频谱产生大量的杂散分量,对SFDR和ENOB造成极坏的影响。\par
	失配误差会严重减少和限制时间交错ADC的性能,所以早期有大量的文献\citeup{vogel2005impact}\citeup{kurosawa2001explicit}\citeup{jenq1988digital}对增益、偏置、时钟失配误差进行了研究。并建立了分时交错ADC等效误差模型,全面分析了各种误差对系统的性能的影响。\par
	然后只有很少一部分的文章对静态非线性误差对性能的影响作出了分析,而且大多只是分析了INL和DNL等静态参数\citeup{ieee2000standard}。文献\cite{simoes1997nonlinearity}揭示了实际上非线性误差同样会影响TIADC系统的动态性能,而且不能用传统的INL的测试方式来进行测量。文献\cite{kurosawa2002channel}阐释了非线性失配的一些特性,然而并没有建立相应的数学模型。\par
	本文的失配误差建模和以前不同的地方在于,建立了一个通用的数学模型,不但包括增益、偏置等研究得比较多的线性误差,而且还考虑进去了非线性模型。\par
	文献\cite{vogel2004analysis}显示了使用非线性滤波器组( nonlinear hybrid filter banks,NHFB)来进行建模的优越性,由此可见,可以将混合滤波器组的概念进行一定的推广,然后得到TIADC系统比较全面的输入输出关系。本节还将阐释产生时钟偏置误差的两种不同的噪声源,以及探究增益、偏置失配误差和非线性失配之间的关系。\par	
	为了分析通道失配误差对性能的影响,首先需要构建单通道的数学模型。\par
	\section{非线性系统的数学表达式}\label{sec:nonlinearTheory}		
		
		如果任意一个连续时间且无记忆的非线性函数$f(x)$,其中x处于[a,b]的区间内,那么它可以用一个单变量的多项式函数$p(x)$来进行近似,并且满足

			\begin{equation}
				\left| {f\left( x \right) - p\left( x \right)} \right| < \varepsilon ,\,\,\,\,x \in \left[ {a,b} \right]
			\end{equation}
			
		其中$\varepsilon $的值可以任意小。
		对于这个多项式函数$p(x)$,可以写作:
			\begin{equation}
				\begin{array}{l}
	p\left( x \right) = {c_0}{x^0} + {c_1}{x^1} +  \cdots  + {c_{K - 2}}{x^{K - 2}} + {c_{K - 1}}{x^{K - 1}}\\
	\,\,\,\,\,\,\,\,\,\,\, = \sum\limits_{k = 0}^{K - 1} {{c_k}{x^k}} 
	\end{array} \label{eq:theorem}
			\end{equation}
			
		因此任意一个连续的函数$f(x)$,并且其自变量处于闭区间[a,b]上,可以统一用一个多项式$p(x)$进行表示,而且可以达到任意的精度。这就是所谓的维尔斯特拉斯逼近定理\citeup{zill2011advanced}。正如这个定理所说,我们可以找到一个多项式$p(x)$来对静态非线性误差进行完美的近似。\par
		当输入的是一个随时间变化的信号的时候,非线性函数的输出也随着时间改变。
		此时式\ref{eq:theorem}可以重写为
			
			\begin{equation}
				\tilde x\left( t \right) = \sum\limits_{k = 0}^{K - 1} {{c_k}{x^k}\left( t \right)}  \label{eq:theoremRewrite}
			\end{equation}
			
		对式\ref{eq:theoremRewrite}进行傅里叶变换:
			
			\begin{equation}
				\tilde X\left( {j\Omega } \right) = \sum\limits_{k = 0}^{K - 1} {{c_k}{X^{\left( {*k} \right)}}\left( {j\Omega } \right)} 
			\end{equation}
			
		其中,${X^{\left( {*k} \right)}}\left( {j\Omega } \right)$表示$(k-1)$阶和自己的卷积,也即:
		
			\begin{equation}
				\begin{array}{l}
	{X^{\left( {*k} \right)}}\left( {j\Omega } \right)\\
	 = \left\{ \begin{array}{l}
	2\pi \delta \left( \Omega  \right),k = 0\\
	X\left( {j\Omega } \right),k = 1\\
	{\left( {\frac{1}{{2\pi }}} \right)^{k - 1}}\underbrace {\left( {X*X* \cdots *X} \right)\left( {j\Omega } \right)}_{\left( {k - 1} \right) - convolutions},k \ge 2
	\end{array} \right.
	\end{array}
			\end{equation}
		这样我们就为连续时间信号通过非线性系统找到时域上和频域上确定的输入输出关系。
	\section{等效误差模型的输入输出关系} \label{sec:modelWithError}
		为了分析通道失配误差对性能的英雄,需要先构造单通道的数学模型。
		TIADC第m通道的数学模型如图\ref{pics/chapter2/mthModel}所示,其中非线性用$f(x)$来进行表示。根据式\ref{eq:theoremRewrite},可以把第m通道的采样序列表示为:
			\begin{equation}
				\begin{array}{l}
{y_m}\left[ n \right] = \sum\limits_{k = 0}^{K - 1} {{c_k}{x^k}\left( t \right)}  = \sum\limits_{k = 0}^{K - 1} {{c_k}{x^k}\left( {nM{T_s} + m{T_s}} \right)} \\
\,\,\,\,\,\,\,\,\,\,\,\, = {o_m} + {g_m}x\left( {nM{T_s} + m{T_s}} \right) + \sum\limits_{k = 2}^{K - 1} {{c_k}{x^k}\left( {nM{T_s} + m{T_s}} \right)} \label{eq:y_m}
\end{array}
			\end{equation}
		% ===========pic==========
			\pic[htbp]{第m通道的数学模型}{}{pics/chapter2/mthModel}
		其中$o_m$、$g_m$分别表示偏置、增益失配误差。\par
		可以看出其实多项式$c_k$的第0阶和第1阶就是偏置误差和增益失配误差。\par
		为了更容易理解误差建模的基本思想,结合图\ref{pics/chapter2/mthModel}、\ref{pics/chapter2/principleOfTIADC}两幅图,可以把图\ref{pics/chapter2/mthModel}中的系统模型进行等效转换,然后得出图\ref{pics/chapter2/NHFBofTIADC}的TIADC等效模型图。
		% ==========pic==========
			\pic[htbp]{TIADC等效误差模型}{}{pics/chapter2/NHFBofTIADC}
			
		
	为了研究每个通道的输出,我们需要对每个通道的每个模块的输出进行进一步的分析。
	\par
	输入信号为${x_m}(t)$,模拟滤波器组为${H_m}(j\Omega )$,是一个延迟器。即
	
		\begin{equation}
			{H_m}(j\Omega ) = {e^{j\Omega (m{T_s} - {r_m}{T_s})}},m = 0,1, \cdot  \cdot  \cdot ,M - 1
		\end{equation}
	那么通过模拟滤波器组以后的信号的傅里叶变换式为:
		\begin{equation}
			{X_m}\left( {j\Omega } \right) = {X_a}\left( {j\Omega } \right){H_m}\left( {j\Omega } \right)
		\end{equation}
		
	模拟信号${x_a}(t)$经过模拟滤波器组以后,需要进行采样。然后紧挨的是非线性误差模块$f_m(x)$以及上采样模块。因为$f_m(x)$是无记忆且非线性的函数,所以加上非线性误差和采样的顺序无关紧要。如图\ref{pics/chapter2/changeOrder}所示。
	
	 %==================pic====================
		\pic[htpb]{加非线性误差和采样模块的顺序可以更改}{}{pics/chapter2/changeOrder}
		
		在\ref{sec:upsample}小节里面曾经说过上采样的
				\begin{equation}
				\begin{array}{l}
Y\left( {{e^{j\Omega {T_s}}}} \right) = X\left( {{e^{j\Omega M{T_s}}}} \right)\\
\,\,\,\,\,\,\,\,\,\,\,\,\,\,\,\,\,\,\, = \frac{1}{{M{T_s}}}\sum\limits_{p =  - \infty }^\infty  {X\left( {{e^{j\left( {\Omega  - p\frac{{{\Omega _s}}}{M}} \right)}}} \right)} 
\end{array} 
			\end{equation}
		由\ref{sec:upsample}中的式\ref{eq:Y_upsample}可以知道,经过上采样以后的序列的傅里叶变换为$Y\left( {{e^{j\Omega {T_s}}}} \right) = \frac{1}{{M{T_s}}}\sum\limits_{p =  - \infty }^\infty  {X\left( {j\left( {\Omega  - p\frac{{{\Omega _s}}}{M}} \right)} \right)} $。同时代入
		\ref{sec:nonlinearTheory}小节中的\ref{eq:theoremRewrite}式子,可以得到采样、加上非线性误差以及上采样以后的序列${{\tilde y}_m}\left( n \right)$的傅里叶变换为
			\begin{equation}
				{\tilde Y_m}\left( {j\Omega M{T_s}} \right) = \frac{1}{{M{T_s}}}\sum\limits_{p =  - \infty }^\infty  {\sum\limits_{k = 0}^{K - 1} {{c_{k,m}}X_m^{\left( {*k} \right)}\left( {j\left( {\Omega  - p\frac{{{\Omega _s}}}{M}} \right)} \right)} }  \label{eq:Y_mWithError}
			\end{equation}
		其中k和表示非线性的多项式的阶数有关;m表示当前处于第m通道,p是采样以后频谱以抽样频率为间隔重复,进行周期延拓的指数;运算符$(*k)$表示进行k-1阶的卷积。\par
		从图\ref{pics/chapter2/NHFBofTIADC}中可以看出TIADC的带失配输出的信号频谱$Y(e^{j\omega })$是系统的M个通道输出信号的合路的傅里叶变换。\ref{sec:nonlinearTheory}小节中说过,输出信号合路等效于每一个通道频谱${Y_m}(e^{j\omega })$进行移位求和,其数学表达式如下:
		 %==================equation====================
			\begin{equation}
				\begin{array}{l}
Y\left( {{e^{j\Omega {T_s}}}} \right) = \sum\limits_{m = 0}^{M - 1} {{Y_m}\left( {{e^{j\Omega {T_s}}}} \right)} \\
\,\,\,\,\,\,\,\,\,\,\,\,\,\,\,\,\,\,\, = \sum\limits_{m = 0}^{M - 1} {{G_m}\left( {{e^{j\Omega {T_s}}}} \right){{\tilde Y}_m}\left( {{e^{j\Omega M{T_s}}}} \right)} 
\end{array}   \label{eq:combiner}
			\end{equation}
		其中${G_m}({e^{j\Omega {T_s}}}) = {e^{ - j\Omega m}}{|_{\omega  = \Omega {T_s}}} = {e^{ - j\Omega {T_s}m}}$,表示移位。
		\par
		
		把式\ref{eq:Y_mWithError}代入式\ref{eq:combiner}中,然后把得到的公式重写如下:
			 %==================equation====================
			 \begin{equation}
				Y\left( {{e^{j\Omega {T_s}}}} \right) = \frac{1}{{{T_s}}}\sum\limits_{k = 0}^{K - 1} {\sum\limits_{p =  - \infty }^\infty  {{T_{p,k}}\left( {j\Omega } \right)} }  \label{eq:YwithError}
			\end{equation}
			
		其中,
			\begin{equation}
				{T_{p,k}}\left( {j\Omega } \right) = \frac{1}{M}\sum\limits_{m = 0}^{M - 1} {{c_{k,m}}} \left| {_{j\Omega  = j\left( {\Omega  - p\frac{{{\Omega _s}}}{M}} \right)}} \right. \times {G_m}\left( {{e^{j\Omega {T_s}}}} \right)\label{eq:TpkFrame}
			\end{equation}
	
	如图\ref{pics/chapter2/analogChannelModel}所示为TIADC第m个通道的模型,此模型中包含了模拟输入行为级表达式、孔径延迟、偏置、增益和非线性失配误差,但是没有考虑量化误差以及时钟抖动的影响。下面分别对这几种误差进行简单的分析:
	\pic[htpb]{TIADC第m个通道的模型}{}{pics/chapter2/analogChannelModel}
		\begin{enumerate}
			\item 输入信号的行为级表达式。\par
				每个通道的输入信号的行为级表达式由采样-保持电路的保持状态决定。如果不考虑采样-保持阶段的非线性作用,我们可以用${P_m(j\Omega )}$来描述其频率响应。
			\item 孔径延迟$\theta _m$。\par
				所谓的孔径延迟指的是时钟信号触发采样的时间和实际真正开始采样的时间之间的差值,也就是说孔径延迟是指在保持命令发出之后到ADC采样保持放大器(SHA)完全打开采样开关所需的时间。
			\item 非线性误差:包含了增益、偏置以及非线性失配误差等等。
		\end{enumerate}
	
	基于以上的分析,下面将依次分析\ref{eq:YwithError}中所有模块具体的框图。\par
	
		\begin{itemize}
			\item 延迟滤波器${H_m}(j\Omega )$:
				\begin{equation}
					{H_m}\left( {j\Omega } \right) = {P_m}\left( {j\Omega } \right){e^{ - j\Omega {\theta _m}}}{e^{j\Omega {T_s}m}} \label{eq:delayFilter}
				\end{equation}
		
		在式\ref{eq:delayFilter}中,我们可以看到我们加入了孔径延迟和时间偏移${T_s}m$。这是TIADC系统的一种等效模型,因为如果在时钟路径上有一点的超前其实就相当于在信号路径上有一点延迟,它们是等价的关系。
		
			\item 合路滤波器$G_m(e^{j\Omega T_s})$:
				\begin{equation}
					G_m(e^{j\Omega T_s})=e^{-j\Omega m} | _{\omega =\Omega T_s} = e^{-j\Omega T_s m}\label{eq:synthesisFilter}
				\end{equation}
		\end{itemize}
		将式\ref{eq:delayFilter}和式\ref{eq:synthesisFilter}带入\ref{eq:TpkFrame}中
		可以得到:
		 %==================equation====================
		 \begin{equation}
			T{ _{p,k}}\left( {j\Omega } \right) = \frac{1}{M}\sum\limits_{m = 0}^{M - 1} {{c_{k,m}} \times {{\left( {{X_a}\left( {j\Omega } \right){P_m}\left( {j\Omega } \right){e^{ - j\Omega {\theta _m}}}{e^{j\Omega {T_s}m}}} \right)}^{\left( {*k} \right)}}} \left| {_{j\Omega  = j\left( {\Omega  - p\frac{{{\Omega _s}}}{M}} \right)}} \right. \times {e^{ - j\Omega {T_s}m}} \label{eq:TpkPm}
		\end{equation}
		
		将输入函数${P_m)(j\Omega )}$分为幅度和相位两部分,即
		 %==================equation====================
		 \begin{equation}
			{P_m}\left( {j\Omega } \right) = {A_m}\left( \Omega  \right){e^{ - j{\tau _m}\Omega }}{e^{j{\phi _m}\left( \Omega  \right)}}
		\end{equation}
		
		其中${e^{ - j{\tau _m}\Omega }}$是相位响应的线性部分,${e^{j{\phi _m}\left( \Omega  \right)}}$是相位响应的非线性部分。然后把\ref{eq:TpkPm}中所有的在卷积中的线性部分都提出来,重新写作:
		 %==================equation====================
		 \begin{equation}
			{T_{p,k}}\left( {j\Omega } \right) = \frac{1}{M}\sum\limits_{m = 0}^{M - 1} {{c_{k,m}}{{\left( {{X_a}\left( {j\Omega } \right){A_m}\left( \Omega  \right){e^{j{\phi _m}\left( \Omega  \right)}}} \right)}^{\left( {*k} \right)}}\left| {_{j\Omega  = j\left( {\Omega  - p\frac{{{\Omega _s}}}{M}} \right)}} \right. \times {e^{ - j\left( {\Omega  - p\frac{{{\Omega _s}}}{M}} \right)\left( {{\theta _m} + {\tau _m}} \right)}}{e^{ - jpm\frac{{2\pi }}{M}}}}  \label{eq:TpkRewrite}
		\end{equation}
		
		可以把$\theta $和$\tau $写成平均值加上偏差的形式,即${\theta _m} = {\theta _R} + \Delta {\theta _m},{\tau _m} = {\tau _R} + \Delta {\tau _m}$。
		代入\ref{eq:TpkRewrite}中可以得到
		 %==================equation====================
		 \begin{equation}
			\begin{array}{ccccc}
			{T_{p,k}}\left( {j\Omega } \right) = \frac{1}{M}{e^{ - j\left( {\Omega  - p\frac{{{\Omega _s}}}{M}} \right)\left( {{\theta _R} + {\tau _R}} \right)}}\sum\limits_{m = 0}^{M - 1} {{c_{k,m}}} {\left( {{X_a}\left( {j\Omega } \right){A_m}\left( \Omega  \right){e^{j{\phi _m}\left( \Omega  \right)}}} \right)^{\left( {*k} \right)}}\left| {_{j\Omega  = j\left( {\Omega  - p\frac{{{\Omega _s}}}{M}} \right)}} \right.\\
			\,\,\,\,\,\,\,\,\,\,\,\,\,\,\,\,\,\,\,\,\,\, \times {e^{ - j\left( {\Omega  - p\frac{{{\Omega _s}}}{M}} \right)\Delta {t_m}}}{e^{ - jpm\frac{{2\pi }}{M}}}
			\end{array}\label{eq:TpkWithTiming}
		\end{equation}
		
		式\ref{eq:TpkWithTiming}中包含了全局的线性移位${e^{ - j\left( {\Omega  - p\frac{{{\Omega _s}}}{M}} \right)\left( {{\theta _R} + {\tau _R}} \right)}}$,而且它与m和k独立。这个全局的线性相移只会造成延迟而不会引入更多地杂散分量,也即它满足完美重构的特性。因此我们可以不再考虑这个线性相移,可以把它移出表达式以简化式子。\par
		为了体现上述包含非线性误差的模型和偏置、增益、时钟失配误差的关系,我们将进行如下的讨论。\par
		从式\ref{eq:y_m}可以看出$o_m$、$g_m$分别表示偏置、增益失配误差。
		可以看出其实多项式$c_k$的第0阶和第1阶就是偏置误差和增益失配误差。所以可以重写式\ref{eq:TpkWithTiming}和\ref{eq:YwithError}:
		 %==================equation====================
		 \begin{equation}
			Y\left( {{e^{j\Omega {T_s}}}} \right) = \frac{1}{{{T_s}}}\sum\limits_{p =  - \infty }^\infty  {\left[ {{\beta _p}2\pi \delta \left( {\Omega  - p\frac{{{\Omega _s}}}{M}} \right) + {\alpha _p}\left( {j\Omega } \right){X_a}\left( {j\left( {\Omega  - p\frac{{{\Omega _s}}}{M}} \right)} \right) + \sum\limits_{k = 2}^{K - 1} {{T_{p,k}}\left( {j\Omega } \right)} } \right]} \label{eq:YwithGainOffset}
		\end{equation}
		其中
		\begin{equation}
			{\beta _p} = \frac{1}{M}\sum\limits_{m = 0}^{M - 1} {{o_m}{e^{ - jpm\frac{{2\pi }}{M}}}} 
		\end{equation}
		
		\begin{equation}
			{\alpha _p}\left( {j\Omega } \right) = \frac{1}{M}\sum\limits_{m = 0}^{M - 1} {{g_m}{A_m}\left( {\Omega  - p\frac{{{\Omega _s}}}{M}} \right) \times {e^{ - j\left( {\Omega  - p\frac{{{\Omega _s}}}{M}} \right)\Delta {t_m}}}{e^{j{\phi _m}\left( {\Omega  - p\frac{{{\Omega _s}}}{M}} \right)}}{e^{ - jpm\frac{{2\pi }}{M}}}} 
		\end{equation}
		
		\begin{equation}
			{T_{p,k}}\left( {j\Omega } \right) = \frac{1}{M}\sum\limits_{m = 0}^{M - 1} {{c_{k,m}}{{\left( {{X_a}\left( {j\Omega } \right){A_m}\left( \Omega  \right){e^{j{\phi _m}\left( \Omega  \right)}}} \right)}^{\left( {*k} \right)}}\left| {_{j\Omega  = j\left( {\Omega  - p\frac{{{\Omega _s}}}{M}} \right)}} \right. \times {e^{ - j\left( {\Omega  - p\frac{{{\Omega _s}}}{M}} \right)\Delta {t_m}}}{e^{ - jpm\frac{{2\pi }}{M}}}} 
		\end{equation}
		
\section{TIADC失配误差分析}
		理论上,TIADC可以通过增加M来提高整体的采样率。但是由于工艺制造技术限制必然会引入通道间的失配误差,严重影响系统的动态性能。在\ref{sec:modelWithError}小节中,我们得到了含有增益误差、偏置误差以及非线性误差的TIADC误差等效模型。本小节将利用上述模型进一步分析各种误差的频谱以及相位响应误差和时钟偏置误差的关系。
	\subsection{偏置失配误差}
		偏置失配误差指的是TIADC通道间直流偏置不一致而引入的误差。直流偏置失配在单通道的ADC中表现为直流分量,这是无法避免的,但是在多通道的A/D转换器中,这样的失配误差会引起谐波失真,从而影响整体的性能。\par
		为了讨论偏置误差对输入输出关系的影响,我们可以令式\ref{eq:YwithGainOffset}中$\Delta t_m = 0$,$o_m=0$以及非线性误差系数$c_{k,m},k \ge 2$为0,可以得到化简以后的公式为:
				 %==================equation====================
		\begin{equation}
			\[Y({e^{j\Omega }}) = \frac{1}{{{T_s}}}\sum\limits_{p =  - \infty }^{ + \infty } {{X_a}\left( {\Omega  - p\frac{{{\Omega _s}}}{M}} \right) + \frac{1}{{{T_s}}}\sum\limits_{p =  - \infty }^{ + \infty } {\left( {\left( {\frac{{2\pi }}{M}\sum\limits_{m = 0}^{M - 1} {{o_m}{e^{ - jp\frac{{2\pi m}}{M}}}} } \right)\delta \left( {\Omega  - p\frac{{{\Omega _s}}}{M}} \right)} \right)} } \]\label{eq:offsetErrors}
		\end{equation}
		式\ref{eq:offsetErrors}可以看作是TIADC理想的输出频谱与偏置失配误差引起的杂散分量之和。因此,偏置失配误差引入的分量在:
		\begin{equation}
			\[\Omega  = p\frac{{{\Omega _s}}}{M},p = 0, \pm 1, \pm 2, \ldots , \pm M - 1\]
		\end{equation}		
	\subsection{增益失配误差}
		增益误差的原因在于通道间的增益不一致。如果只是考虑单通道的话,增益误差不引起严重的后果,但是在TIADC这种多通道的系统中,增益失配相当于对输入信号进行幅度调制,所以会引起信号的失真。
		首先令公式\ref{eq:YwithGainOffset}中的$\Delta t_m=0$,$o_m=0$以及非线性系数为0,然后式\ref{eq:YwithGainOffset}可以化作:
		 %==================equation====================
		\begin{equation}
			\[\begin{array}{l}
			Y({e^{j\Omega }}) = \frac{1}{{{T_s}}}\sum\limits_{p =  - \infty }^{ + \infty } {{X_a}\left( {\Omega  - p\frac{{{\Omega _s}}}{M}} \right) + } \frac{1}{{{T_s}}}\sum\limits_{p =  - \infty }^{ + \infty } {\left( {\frac{1}{M}\sum\limits_{m = 0}^{M - 1} {({g_m} - 1){e^{ - jp\frac{{2\pi m}}{M}}}} } \right){X_a}\left( {\Omega  - p\frac{{{\Omega _s}}}{M}} \right)} \\
		= \frac{1}{{{T_s}}}\sum\limits_{p =  - \infty }^{ + \infty } {{X_a}\left( {\Omega  - p\frac{{{\Omega _s}}}{M}} \right) + } {E_g}\left( {{e^{j\Omega }}} \right)
		\end{array}\]\label{eq:gainErrors}
		\end{equation}
		
		式\ref{eq:gainErrors}同样可以看作理想情况的频域表达式和杂散分量之和,记杂散分量为${E_g}(e^{j\Omega })$。\par
		可以看出杂散频谱分量与输入信号有关,可以以正弦信号$x(t)=\sin(\Omega _in t)$作为输入信号(其频域表达式为$\[X(j\Omega ) = j\pi [\delta (\Omega  + {\Omega _{in}}) - \delta (\Omega  - {\Omega _{in}})]\]$),可以把${E_g}(e^{j\Omega })$化为:
		 %==================equation====================
		 \begin{equation}
			\[{E_g}\left( {{e^{j\Omega }}} \right) = \frac{{j\pi }}{{{T_s}}}\sum\limits_{p =  - \infty }^{ + \infty } {\left( {\frac{1}{M}\sum\limits_{m = 0}^{M - 1} {({g_m} - 1){e^{ - jp\frac{{2\pi m}}{M}}}} } \right)[\delta (\Omega  + {\Omega _{in}} - p\frac{{{\Omega _s}}}{M}) - \delta (\Omega  - {\Omega _{in}} - p\frac{{{\Omega _s}}}{M})]} \]
		\end{equation}
		
		可以看出增益失配引入的杂散分量处于如下的频点:
		 %==================equation====================
		 \begin{equation}
			\[\Omega  =  \pm {\Omega _{in}} + p\frac{{{\Omega _s}}}{M},p = 0, \pm 1, \pm 2, \ldots , \pm M - 1\]
		\end{equation}
	
	 %==================subsection====================
	 \subsection{时钟失配误差}
		时钟失配误差指的是通道中采样的时间间隔不完全相同,使得TIADC的均匀采样变成了非均匀采样,因而引入的误差。引起时间误差的因素将在\ref{sec:sourceOfTiming}小节中详细的讨论。\par
		
		为了分析时钟失配误差带来的影响,可以把其他的误差都置为0。于是可以将式\ref{eq:YwithGainOffset}化做:
		 %==================equation====================
		 \begin{equation}
			\[Y({e^{j\Omega }}) = \frac{1}{{{T_s}}}\sum\limits_{p =  - \infty }^{ + \infty } {{X_a}\left( {j\frac{\Omega }{{{T_s}}} - jp\frac{{2\pi }}{{M{T_s}}}} \right)} \frac{1}{M}\sum\limits_{m = 0}^{M - 1} {{e^{ - jp\frac{{2\pi m}}{M}}}{e^{j(\Omega  - p\frac{{2\pi }}{M})\Delta {t_m}}}} \]
		\end{equation}
		
		观察式子,可以发现时钟失配误差也和输入信号相关。同样输入正弦信号,得到输出的表达式为:
		 %==================equation====================
		 \begin{equation}
			\[Y({e^{j\Omega }}) = \frac{{j\pi }}{{{T_s}}}\sum\limits_{p =  - \infty }^{ + \infty } {\left( {\frac{1}{M}\sum\limits_{m = 0}^{M - 1} {{e^{ - jp\frac{{2\pi m}}{M}}}{e^{j(\Omega {T_s} - p\frac{{2\pi }}{M})\Delta {t_m}}}} } \right)\left( {\delta \left( {\Omega  + {\Omega _{in}} - \frac{p}{M}{\Omega _s}} \right) - \delta \left( {\Omega  - {\Omega _{in}} - \frac{p}{M}{\Omega _s}} \right)} \right)} \]
		\end{equation}
		
		同理可以得到时钟失配引入的杂散分量的频点:
			 %==================equation====================
		 \begin{equation}
			\[\Omega  =  \pm {\Omega _{in}} + p\frac{{{\Omega _s}}}{M},p = 0, \pm 1, \pm 2, \ldots , \pm M - 1\]
		\end{equation}
	 %==================subsubsection====================	
		\subsubsection{引起时钟失配的因素}\label{sec:sourceOfTiming}
			时钟失配误差严重影响时间交错ADC的整体的性能,引起时钟失配的主要原因有两种,下面将就时钟抖动和相位响应偏差这两种原因来引起时钟失配的因素进行讨论。\par
			\begin{enumerate}
				\item 时钟抖动。\par
					时钟抖动指的是在采样的时候,如果时钟一直不稳定,特别是在时钟的边沿处发生变化,会导致采样不准确,使得模拟输入信号的实际采样值有微小的误差。\par
					文献\cite{jiao2013estimation}指出,抖动会降低性能,特别是当输入信号频率比较大的时候,转换误差也相应比较大,使得SNR下降。\par
					因为时钟抖动是随机的,只能尽量的减小,而不可能完全消除。降低时钟抖动的方法包括改进时钟源、消抖、滤波、分频等等。\par
				\item 相位响应偏差。\par
					相位响应偏差依赖于输入频率,可以分为和频率成线性关系的线性相位失配和与频率成非线性关系的非线性关系相位失配。线性相位偏差造成的后果在时域上是一个固定的时移,然而非线性相位失配将会产生时域上依赖于输入频率的移位。 		因此线性部分的影响会随着频率增加而增加,但是非线性相位则不然,故相位的失配将严重影响TIADC的整体性能,只有在这种误差引起的信噪比小于一定的量的时候才不会造成毁灭性的影响。但是,因为ADC的线性相位传输特性的存在,所以线性相位偏差比非线性误差的影响更加严重。\par
					
				从图\ref{pics/chapter2/analogChannelModel}可以看出相位失配对时钟误差的影响主要在两方面:孔径延迟$\Delta  \theta _m$和时钟偏差$\Delta \tau _m$。如果只是从输出来看,我们完全无法区分这两种误差源,所以大多数的建模都是把它考虑成同一个东西。
				\begin{enumerate}
					\item 孔径延迟:如图\ref{pics/chapter3/apentureDelay}。所谓的孔径延迟指的是时钟信号触发采样的时间和实际真正开始采样的时间之间的差值,也就是说孔径延迟是指在保持命令发出之后到ADC采样保持放大器(SHA)完全打开采样开关所需的时间。孔径延迟会造成时钟信号的偏移,因此采样会发生在偏移的采样时刻,故我们得到的采样值并不是均匀采样得到的理想值。
						 %==================pic====================
						 \pic[htbp]{孔径延迟的示意图}{}{pics/chapter3/apentureDelay}
					\item 线性相位输入失配:如图\ref{pics/chapter3/linearPhaseMismatch}。这种失配会造成自身信号的偏移,因此,即使我们在理想的采样时刻进行采样,我们一样会获得错误的采样值。
					 %==================pic====================
					 \pic[htbp]{线性相位输入失配}{}{pics/chapter3/linearPhaseMismatch}
				\end{enumerate}
					实际上,我们在建模的时候一般会同时考虑两种的效应的影响。如图\ref{pics/chapter3/sourceOfTiming} 所示,孔径延迟和线性相位输入失配都是时钟失配的来源。
					 %==================pic====================
					 \pic[htbp]{相位响应失配}{}{pics/chapter3/sourceOfTiming}
				\end{enumerate}
		
		
	 %==================subsection====================
	 \subsection{非线性失配误差}
	 同样把其他的失配误差都置为0以后,代入正弦信号,然后提取杂散分量,可以得到如下表达式:
	  %==================equation====================
	  \begin{equation}
		\[\begin{array}{l}
{T_{p,k}}\left( {j\Omega } \right) = \frac{1}{M}\sum\limits_{m = 0}^{M - 1} {{c_{k,m}}} {\left( {\frac{A}{{2j}}\left[ {\delta \left( {\Omega  - {\Omega _0}} \right) - \delta \left( {\Omega  + {\Omega _0}} \right)} \right]{A_m}\left( \Omega  \right){e^{j{\phi _m}\left( \Omega  \right)}}} \right)^{\left( {*k} \right)}}\left| {_{j\Omega  = j\left( {\Omega  - p\frac{{{\Omega _s}}}{M}} \right)}} \right.\\
 \times {e^{ - j\left( {\Omega  - p\frac{{{\Omega _s}}}{M}} \right)\Delta {t_m}}}{e^{ - jpm\frac{{2\pi }}{M}}}
\end{array}\]\label{eq:TpkForLocation}
	\end{equation}
	从式\ref{eq:TpkForLocation}可以看出,如果k>1,也就是我们考虑更加高阶的非线性误差的话,每一个$c_{k,m}$会产生一个杂散分量,考虑到式\ref{eq:TpkForLocation}中的(k-1)阶的卷积,可以得到非线性误差引入的频点位置为:
	 %==================equation====================
	 \begin{equation}
		\[\Omega  =  \pm k{\Omega _{in}} + p\frac{{{\Omega _s}}}{M},p = 0, \pm 1, \pm 2, \ldots , \pm M - 1;k = 0,1,2, \ldots  \ldots \]
	\end{equation}
	
	
	
	
	 
	
